\documentclass{article}%options: 11pt, A4paper, 

\usepackage{amssymb,amsfonts,amsmath,amsthm} 
\usepackage[left=1.75in,right=1.75in,bottom=1.5in,top=1in]{geometry}%options:
%\usepackage{fontspec}
%\usepackage{xunicode}
%\usepackage{xltxtra}
%\usepackage{textpos,multicol,everyshi,rotating}%for positioning text blocks. See XeLaTeXposter for examples
%\usepackage[clock,weather,alpine]{ifsym}
%\setlength{\parindent}{0pt}
%\usepackage{pstricks}

%%%%%%%     PACKAGES      %%%%%%% 
\usepackage{colortbl}
\usepackage{enumerate}
\usepackage{graphicx}
\usepackage{xcolor}
%\usepackage{verse}
%\usepackage{yfonts}

%%%%%%%  XY PIC COMMANDS  %%%%%%%

\usepackage{xy} 
\xyoption{all} 
\newdir{ >}{{}*!/-12pt/@{>}} 
\newdir{ -}{{}*!/-5pt/@{}} 
%\newdir{- }{{}*!/-5pt/@{}} 
\newdir{- }{{}*!/-10pt/@{}} 
\newdir{ ^(}{{}*!/-5pt/@^{(}} 
\newdir{ _(}{{}*!/-5pt/@_{(}}
%\newdir{ |}{{}*!/-5pt/@{|}}
\newdir{ |}{{}*!/-10pt/@{|}}
\newdir{_(}{{}*!/0pt/@_{(}}
\xyoption{arc}
\xyoption{ps}

%%%%%%%  PS TRICKS         %%%%%%%
\usepackage{pstricks}

%%%%%%%  AMS THEOREM TAGS  %%%%%%%
\theoremstyle{plain} 

\newtheorem{hilfssatz}{Hilfssatz}
\newtheorem{conjecture}{Conjecture} 
\newtheorem{question}{Question} 
\newtheorem{corollary}{Corollary} 
\newtheorem{lemma}{Lemma} 
\newtheorem{proposition}{Proposition} 
\newtheorem{algorithm}{Algorithm} 
\newtheorem*{exercise}{Exercise} 
\newtheorem{aufgabe}{Aufgabe} 
\newtheorem{prop}{Proposition}

\theoremstyle{remark} 
\newtheorem*{prf}{Proof} 
\newtheorem*{beweis}{Beweis}
\newtheorem{remark}{Remark} 
\newtheorem*{remarks}{Remarks} 
\newtheorem*{solution}{Solution} 
\newtheorem{Bemerkung}{Bemerkung}
\newtheorem{observation}{Observation}

\newtheorem*{losung}{L"osung}
\newtheorem*{lösung}{Lösung}

\theoremstyle{definition} 
\newtheorem{definition}{Definition} 
\newtheorem{theorem}{Theorem}
\newtheorem{example}{Example} 
\newtheorem{notation}{Notation}
\newtheorem{Aufgabe}{Aufgabe}
\newtheorem{convention}{Convention}
%%%%%%%  MY MATH OPERATORS  %%%%%%%

	%% Set Theoretic / Functions
\DeclareMathOperator{\im}{im} 		
		%% Image
\DeclareMathOperator{\id}{id} 
		%% identity map 
\DeclareMathOperator{\card}{card} 
		%% Cardinality of a set 

	%% Algebraic Operators
\DeclareMathOperator{\Irr}{Irr} 
		% Subalgebra of irreducible elts
\DeclareMathOperator{\Red}{Red} 
		% Subalgebra of reducible elts
\DeclareMathOperator{\Hom}{Hom} 
		% Collection of homomorphisms 
\DeclareMathOperator{\SL}{SL} 
		% Special Linear group
\DeclareMathOperator{\GL}{GL} 
		% General Linear group
\DeclareMathOperator{\wt}{wt} 
		% weight
\DeclareMathOperator{\Symb}{Symb}
		% Symbolic algebra
\DeclareMathOperator{\Span}{span}
		% Vector space spanning set
\DeclareMathOperator{\Gal}{Gal}
		% Galois group
\DeclareMathOperator{\Char}{char}
		% Characteristic of a Ring
\DeclareMathOperator{\Vect}{\underline{Vect}}

\DeclareMathOperator{\Aut}{Aut}

\DeclareMathOperator{\End}{End}
		% Automorphism Group
\DeclareMathOperator{\lcm}{lcm}
		% Least Common Multiple
\DeclareMathOperator{\mult}{mult}
		% Multiplicity
\DeclareMathOperator{\Tor}{Tor}
\DeclareMathOperator{\tor}{Tor}
		% Homology of right derived functor of tensor product
\DeclareMathOperator{\trdeg}{tr\ deg}
		% Transcendence Degree
\DeclareMathOperator{\krdim}{kr\ dim}
		% Krull Dimension
\DeclareMathOperator{\amdim}{AM\ dim}
		% Atiyah Macdonald dimension
\DeclareMathOperator{\height}{ht}
		% height of prime ideal
\DeclareMathOperator{\amd}{AM\ d}
		% Atiyah Macdonald degree
\newcommand{\amdel}{\text{AM }\delta}
		% Atiyah Macdonald delta
\DeclareMathOperator{\ord}{ord}
		% Order
\DeclareMathOperator{\Open}{Open}
		% The Open functor.
\DeclareMathOperator{\Ab}{\underline{Ab}}
\DeclareMathOperator{\Man}{\underline{Man}}
\DeclareMathOperator{\Haus}{\underline{Haus}}
		% Category of Abelian Groups
\DeclareMathOperator{\Mod}{\underline{Mod}}
		% The category of Modules (left, right, or commutative)		
\DeclareMathOperator{\Set}{\underline{Set}}
		% Set
\DeclareMathOperator{\Gp}{\underline{Gp}}
\DeclareMathOperator{\Rng}{\underline{Rng}}
\DeclareMathOperator{\Top}{\underline{Top}}
\DeclareMathOperator{\SHot}{\underline{SHot}}
\DeclareMathOperator{\hTop}{\underline{hTop}}
\DeclareMathOperator{\Hot}{\underline{Hot}}
\DeclareMathOperator{\LeftMod}{{\phantom{\Mod}\hspace{-20pt}}_{Λ}{\kern -2pt}\Mod}
		% left mods
\DeclareMathOperator{\LeftModPrime}{{\phantom{\Mod}\hspace{-20pt}}_{\Lambda'}{\kern -2pt}\Mod}
\DeclareMathOperator{\Cat}{\underline{Cat}}

\DeclareMathOperator{\Ext}{Ext}
		% The right derived functor of $\Hom$
\DeclareMathOperator{\DirectLim}{\underrightarrow{\lim}}
\DeclareMathOperator{\colim}{colim}
\DeclareMathOperator{\coker}{coker}
\DeclareMathOperator{\gp}{gp}
\DeclareMathOperator{\dom}{dom}
\DeclareMathOperator{\cod}{cod}
\DeclareMathOperator{\sgn}{sgn}
\DeclareMathOperator{\Sympl}{Sympl}

	%%Topology Operators
\DeclareMathOperator{\ofK}{\mathbf{ofK}} 
		% offene Kern
\DeclareMathOperator{\abH}{\mathbf{abH}} 
		% abgeschlossene H"ulle
\DeclareMathOperator{\Ran}{\mathbf{Ran}}
		% Rand
\DeclareMathOperator{\Int}{\mathbf{Int}}
		% Interior
\DeclareMathOperator{\Cls}{\mathbf{Cls}}
		% Closure
\DeclareMathOperator{\Bdy}{\mathbf{Bdy}}
		% Boundary
\DeclareMathOperator{\Isom}{Isom}
\DeclareMathOperator{\Transl}{Transl}
\DeclareMathOperator{\supp}{supp}
\DeclareMathOperator{\rk}{rk}
\DeclareMathOperator{\Ann}{Ann}
\DeclareMathOperator{\length}{length}
\DeclareMathOperator{\Alg}{\underline{Alg}}
%%%%%%%  NEW COMMANDS  %%%%%%%
\DeclareMathOperator{\arcsec}{arcsec}
\DeclareMathOperator{\arccot}{arccot}
\DeclareMathOperator{\arccsc}{arccsc}
\newcommand{\st}{ \, \vert \, }
\newcommand{\paren}[2]{ \left( #1 \right) } 
\newcommand{\surjectivearrow}{\twoheadrightarrow}
\newcommand{\bbCP}{\mathbb{CP}} 
\newcommand{\bbRP}{\mathbb{RP}} 

\newcommand{\der}{\mathrm{d}}
\newcommand{\normal}{\vartriangleleft}

%%%%%%%  FONT ABBREvS  %%%%%%%
\newcommand{\bbA}{\mathbb{A}} 
\newcommand{\bbB}{\mathbb{B}}
\newcommand{\bbC}{\mathbb{C}} 
\newcommand{\bbD}{\mathbb{D}}
\newcommand{\bbE}{\mathbb{E}}
\newcommand{\bbF}{\mathbb{F}} 
\newcommand{\bbG}{\mathbb{G}}
\newcommand{\bbH}{\mathbb{H}}
\newcommand{\bbI}{\mathbb{I}} 
\newcommand{\bbJ}{\mathbb{J}}
\newcommand{\bbK}{\mathbb{K}}
\newcommand{\bbL}{\mathbb{L}} 
\newcommand{\bbM}{\mathbb{M}}
\newcommand{\bbN}{\mathbb{N}}
\newcommand{\bbO}{\mathbb{O}} 
\newcommand{\bbP}{\mathbb{P}}
\newcommand{\bbQ}{\mathbb{Q}}
\newcommand{\bbR}{\mathbb{R}} 
\newcommand{\bbS}{\mathbb{S}}
\newcommand{\bbT}{\mathbb{T}}
\newcommand{\bbU}{\mathbb{U}} 
\newcommand{\bbV}{\mathbb{V}}
\newcommand{\bbW}{\mathbb{W}} 
\newcommand{\bbX}{\mathbb{X}}
\newcommand{\bbY}{\mathbb{Y}}
\newcommand{\bbZ}{\mathbb{Z}}

\newcommand{\calA}{\mathcal{A}} 
\newcommand{\calB}{\mathcal{B}}
\newcommand{\calC}{\mathcal{C}} 
\newcommand{\calD}{\mathcal{D}}
\newcommand{\calE}{\mathcal{E}}
\newcommand{\calF}{\mathcal{F}} 
\newcommand{\calG}{\mathcal{G}}
\newcommand{\calH}{\mathcal{H}}
\newcommand{\calI}{\mathcal{I}} 
\newcommand{\calJ}{\mathcal{J}}
\newcommand{\calK}{\mathcal{K}}
\newcommand{\calL}{\mathcal{L}} 
\newcommand{\calM}{\mathcal{M}}
\newcommand{\calN}{\mathcal{N}}
\newcommand{\calO}{\mathcal{O}} 
\newcommand{\calP}{\mathcal{P}}
\newcommand{\calQ}{\mathcal{Q}}
\newcommand{\calR}{\mathcal{R}} 
\newcommand{\calS}{\mathcal{S}}
\newcommand{\calT}{\mathcal{T}}
\newcommand{\calU}{\mathcal{U}} 
\newcommand{\calV}{\mathcal{V}}
\newcommand{\calW}{\mathcal{W}} 
\newcommand{\calX}{\mathcal{X}}
\newcommand{\calY}{\mathcal{Y}}
\newcommand{\calZ}{\mathcal{Z}}

\newcommand{\cala}{\mathcal{a}} 
\newcommand{\calb}{\mathcal{b}}
\newcommand{\calc}{\mathcal{c}} 
\newcommand{\cald}{\mathcal{d}}
\newcommand{\cale}{\mathcal{e}}
\newcommand{\calf}{\mathcal{f}} 
\newcommand{\calg}{\mathcal{g}}
\newcommand{\calh}{\mathcal{h}}
\newcommand{\cali}{\mathcal{i}} 
\newcommand{\calj}{\mathcal{j}}
\newcommand{\calk}{\mathcal{k}}
\newcommand{\call}{\ell} 
\newcommand{\calm}{\mathcal{m}}
\newcommand{\caln}{\mathcal{n}}
\newcommand{\calo}{\mathcal{o}} 
\newcommand{\calp}{\mathcal{p}}
\newcommand{\calq}{\mathcal{q}}
\newcommand{\calr}{\mathcal{r}} 
\newcommand{\cals}{\mathcal{s}}
\newcommand{\calt}{\mathcal{t}}
\newcommand{\calu}{\mathcal{u}} 
\newcommand{\calv}{\mathcal{v}}
\newcommand{\calw}{\mathcal{w}} 
\newcommand{\calx}{\mathcal{x}}
\newcommand{\caly}{\mathcal{y}}
\newcommand{\calz}{\mathcal{z}}

\newcommand{\frakA}{\mathfrak{A}} 
\newcommand{\frakB}{\mathfrak{B}}
\newcommand{\frakC}{\mathfrak{C}} 
\newcommand{\frakD}{\mathfrak{D}}
\newcommand{\frakE}{\mathfrak{E}}
\newcommand{\frakF}{\mathfrak{F}} 
\newcommand{\frakG}{\mathfrak{G}}
\newcommand{\frakH}{\mathfrak{H}}
\newcommand{\frakI}{\mathfrak{I}} 
\newcommand{\frakJ}{\mathfrak{J}}
\newcommand{\frakK}{\mathfrak{K}}
\newcommand{\frakL}{\mathfrak{L}} 
\newcommand{\frakM}{\mathfrak{M}}
\newcommand{\frakN}{\mathfrak{N}}
\newcommand{\frakO}{\mathfrak{O}} 
\newcommand{\frakP}{\mathfrak{P}}
\newcommand{\frakQ}{\mathfrak{Q}}
\newcommand{\frakR}{\mathfrak{R}} 
\newcommand{\frakS}{\mathfrak{S}}
\newcommand{\frakT}{\mathfrak{T}}
\newcommand{\frakU}{\mathfrak{U}} 
\newcommand{\frakV}{\mathfrak{V}}
\newcommand{\frakW}{\mathfrak{W}} 
\newcommand{\frakX}{\mathfrak{X}}
\newcommand{\frakY}{\mathfrak{Y}}
\newcommand{\frakZ}{\mathfrak{Z}}

\newcommand{\fraka}{\mathfrak{a}} 
\newcommand{\frakb}{\mathfrak{b}}
\newcommand{\frakc}{\mathfrak{c}} 
\newcommand{\frakd}{\mathfrak{d}}
\newcommand{\frake}{\mathfrak{e}}
\newcommand{\frakf}{\mathfrak{f}} 
\newcommand{\frakg}{\mathfrak{g}}
\newcommand{\frakh}{\mathfrak{h}}
\newcommand{\fraki}{\mathfrak{i}} 
\newcommand{\frakj}{\mathfrak{j}}
\newcommand{\frakk}{\mathfrak{k}}
\newcommand{\frakl}{\mathfrak{l}} 
\newcommand{\frakm}{\mathfrak{m}}
\newcommand{\frakn}{\mathfrak{n}}
\newcommand{\frako}{\mathfrak{o}} 
\newcommand{\frakp}{\mathfrak{p}}
\newcommand{\frakq}{\mathfrak{q}}
\newcommand{\frakr}{\mathfrak{r}} 
\newcommand{\fraks}{\mathfrak{s}}
\newcommand{\frakt}{\mathfrak{t}}
\newcommand{\fraku}{\mathfrak{u}} 
\newcommand{\frakv}{\mathfrak{v}}
\newcommand{\frakw}{\mathfrak{w}} 
\newcommand{\frakx}{\mathfrak{x}}
\newcommand{\fraky}{\mathfrak{y}}
\newcommand{\frakz}{\mathfrak{z}}

\newcommand{\rmA}{\textrm{A}} 
\newcommand{\rmB}{\textrm{B}}
\newcommand{\rmC}{\textrm{C}} 
\newcommand{\rmD}{\textrm{D}}
\newcommand{\rmE}{\textrm{E}}
\newcommand{\rmF}{\textrm{F}} 
\newcommand{\rmG}{\textrm{G}}
\newcommand{\rmH}{\textrm{H}}
\newcommand{\rmI}{\textrm{I}} 
\newcommand{\rmJ}{\textrm{J}}
\newcommand{\rmK}{\textrm{K}}
\newcommand{\rmL}{\textrm{L}} 
\newcommand{\rmM}{\textrm{M}}
\newcommand{\rmN}{\textrm{N}}
\newcommand{\rmO}{\textrm{O}} 
\newcommand{\rmP}{\textrm{P}}
\newcommand{\rmQ}{\textrm{Q}}
\newcommand{\rmR}{\textrm{R}} 
\newcommand{\rmS}{\textrm{S}}
\newcommand{\rmT}{\textrm{T}}
\newcommand{\rmU}{\textrm{U}} 
\newcommand{\rmV}{\textrm{V}}
\newcommand{\rmW}{\textrm{W}} 
\newcommand{\rmX}{\textrm{X}}
\newcommand{\rmY}{\textrm{Y}}
\newcommand{\rmZ}{\textrm{Z}}

\newcommand{\Mu}{\textrm{M}}
\newcommand{\Tau}{\textrm{T}}
%%%%%%%%Spacing%%%%%%%%%%%%%%%%%%%
\newcommand{\tab}{\hspace{3ex}}

%\input{/home/glen/Documents/LaTeXDocuments/Fonts}

\DeclareMathOperator{\Fr}{\underline{Frö}}
\DeclareMathOperator{\SmMan}{\underline{SmMan}}
\DeclareMathOperator{\AnMan}{\calA\underline{Man}}

\newcounter{Section}
\newcounter{Objects}
\setcounter{Section}{-1}
\xyoption{rotate}

% Alter some LaTeX defaults for better treatment of figures:
    % See p.105 of "TeX Unbound" for suggested values.
    % See pp. 199-200 of Lamport's "LaTeX" book for details.
    %   General parameters, for ALL pages:
    \renewcommand{\topfraction}{0.9}	% max fraction of floats at top
    \renewcommand{\bottomfraction}{0.8}	% max fraction of floats at bottom
    %   Parameters for TEXT pages (not float pages):
    \setcounter{topnumber}{2}
    \setcounter{bottomnumber}{2}
    \setcounter{totalnumber}{4}     % 2 may work better
    \setcounter{dbltopnumber}{2}    % for 2-column pages
    \renewcommand{\dbltopfraction}{0.9}	% fit big float above 2-col. text
    \renewcommand{\textfraction}{0.07}	% allow minimal text w. figs
    %   Parameters for FLOAT pages (not text pages):
    \renewcommand{\floatpagefraction}{0.7}	% require fuller float pages
	% N.B.: floatpagefraction MUST be less than topfraction !!
    \renewcommand{\dblfloatpagefraction}{0.7}	% require fuller float pages

	% remember to use [htp] or [htpb] for placement


\begin{document}

\begin{center}
{\Large \sc Outline of Algebraic Cobordism Course} 

\bigskip

\end{center}

\section{Spectra, presheaves, formal group laws}

\subsection{Stable homotopy category}
Want: a nice stable homotopy category, whatever that is! The idea, I
believe, is that in the stable homotopy category, call it $\SHot$, one
should have a way of realizing spaces $X, Y \in \Top$ in $\SHot$, call
the realizations $\Sigma^{\infty}X$, so that $f,g : X \rightarrow Y$
which are homotopic then satisfy $\Sigma^{\infty}f = \Sigma^{\infty}g$
and there should be an operation $\Sigma$ extending suspension on
$\Top$ so that $\SHot(\Sigma^{\infty}X, \Sigma^{\infty}Y)\cong
\SHot(\Sigma^{\infty}\Sigma X,\Sigma^{\infty}\Sigma Y)$. Furthermore,
the extended suspension operation should have an inverse $\Omega$, so
that we see $\Sigma$ is an equivalence of categories. We would like
the ``universal'' category with these properties I suppose.

Does such a category exist? What kind of description does it have? How
do we work with it?  For general $\bbX,\bbY\in\SHot$, what does
$\SHot(\bbX,\bbY)$ mean?  Can we relate this back to the regular
homotopy category? Such a category exists for any category with a
``suspension'' functor. See Heller below. 

Adams gives a treatment of it using spectra below. 

\begin{enumerate}
\item Adams, {\it Stable Homotopy and Generalised Homology}
\begin{enumerate}[i. ]
\item Works with Whitehead spectra and $\Omega$-spectra.
\item Defines homotopy groups of spectra, long exact sequence of
  homotopy groups of a spectrum pair.
\item A cofinal subspectrum $E' \subset E$ is a subspectrum
  (structure maps required to be homeomorphisms) for which given any
  $K\subseteq E_n$ a subcomplex, there exists an $m$ such that the
  image of $\Sigma^m K \rightarrow \cdots \rightarrow E_{n+m}$ winds
  up in $E'_{n+m}$.
\item Approaches smash product of spectra by defining ``functions'' of
  spectra, and then defines ``maps'' of spectra as equivalences classes
  of function on cofinal subspectra which agree on their intersection
  or some shared cofinal cofinal subspectrum. ``cells now--maps
  later''. 
\item Then he defines "morphism" of spectra by modding out by a
  homotopy relation.
\item Maps of degree other than 0 are allowed.
\item Prop 2.8 on p. 145 gives a description for $[E,F]_r$ in the
  homotopy category of spectra in terms of the expected notion when
  $E=\Sigma^{\infty}X$.
\item pp 152--153 detail that suspension is an equivalence and how to
  see $[X,Y]$ is an abelian group.
\item The homotopy category of spectra is additive, there are
  cofibers, and a cofibration sequence. Prop 3.9 p. 155. There is a
  fibration sequence.
\item What is meant by Thm 3.12? 
\item The smash construction is done in his stable category. It is
  commutative, associative, has unit, up to coherent isom. in the
  stable category.
\item Smash product $X\wedge Y$ is messy. Essentially choosing a
  ``diagonal'' in the 1st quadrant lattice $X_n \wedge Y_m$. He calls a
  particular choice a ``handicrafted smash product'', and would like to
  show they are equivalent.
\item 30 pages of technical details getting handicrafted smash
  products to work out.
\end{enumerate}

\item Bousfield; Friedlander, {\it Homotopy theory of $\Gamma$-spaces,
  spectra, and bisimplicial sets.}

Also discuss spectra and the stable homotopy category, more with a
view of closed model categories.

\item Heller, {\it Stable Homotopy Categories.}

\begin{enumerate}
\item ``Categories with suspension'', $(\calC, \Sigma)$,
  $\Sigma:\calC\rightarrow\calC$ functor.
\item stable and weakly stable functors of categories with
  suspension. Let $(\calC,\Sigma)$ and $(\calC',\Sigma)$ be CwS,
  $F:\calC\rightarrow \calC'$ functor. It is said to be stable if
  $F\Sigma = \Sigma F : \calC \rightarrow \calC'$ and is said to be
  weakly stable if there is a natural equivalence $\theta : F \Sigma
  \rightarrow \Sigma F$. 
\item A CwS is said to be stable if $\Sigma : \calC \rightarrow \calC$
  is an automorphism, or equivalence of categories.
\item To any CwS $(\calC,\Sigma)$ there is a universal stable CwS
  associated to $\calC$, which we call $s\calC$. There is a weakly
  stable morphism $S:\calC\rightarrow s\calC$ with $\sigma : \Sigma S
  \rightarrow S \Sigma$. It enjoys the following universal property:
  \begin{proposition}
    If $\calA$ is a stable category, $F:\calC\rightarrow \calA $ is
    weakly stable with $\theta : F\Sigma \rightarrow \Sigma F$, then
    there is a unique functor $G: s\calC \rightarrow \calA$ with
    $G\circ S = F$, and $G\sigma = \theta$
  \end{proposition}
\end{enumerate}

In our stable homotopy category, we would like to be able to talk
about fibrations, cofibrations, long exact sequences, etc. We also
would like to somehow realize that $\SHot(X,Y)$ is not just a set, but
an abelian group. So we would like our category to be additive (in the
hom. alg. sense) and triangulated.

We would also like to be able to use the objects of our stable
homotopy category to define homology and cohomology theories.

Relation between $s\hTop$ and the category of symmetric spectra?
Relation with symmetric sequences of Hovey? 

{\bf Heller on homology theories} A homology theory is a functor $h :
\frakC \rightarrow \calA$ with a natural transformation $\partial$
where $\calA$ is a stable abelian category, and $\frakC$ has the
following properties:
\begin{enumerate}
\item it is a category with suspension (suggests stability will be
  involved somehow)
\item it has a homotopy relation (suggests model category)
\item it has a collection of triangles (have cofibration sequence)
\end{enumerate}

\item Hovey; Shipley; Smith, {\it Symmetric Spectra.}

\item Hovey; Palmieri; Strickland, {\it Axiomatic Stable Homotopy Theory.}

\item Puppe, {\it Stabile Homotopietheorie I.}

\item Stong, {\it Notes on Cobordism Theory.}

\item Weibel, {\it An Introduction to Homological Algebra.}

  How does the perspective here relate?

\item Whitehead, {\it Generalized Homology Theories.} 

  Introduces spectra and $\Omega$-spectra. Works only with CW
  complexes and topological categories. Works out many properties and
  theorems specific to this case.

\item Brown, {\it Cohomology Theories.} 

  Contains the celebrated Brown representability theorem. 


\end{enumerate}

\subsection{Formal Group Laws}
Adams, {\it Stable Homotopy and Generalised Homology} 

\begin{definition}
A commutative formal group law is a ring $R$ with a ``multiplication''
$\mu(x,y)\in R[[x,y]]$ which satisfies:
\begin{enumerate}
  \item $\mu(x,0)=x$;
  \item $\mu(0,y)=y$;
  \item $\mu(x,y)=\mu(y,x)$;
  \item $\mu(\mu(x,y),z)=\mu(x,\mu(y,z))$.
\end{enumerate}


A morphism of formal group laws $f:(R,\mu)\rightarrow (S,\nu)$ is a
ring homomorphism $f: R \rightarrow S$ such that $f_*\mu = \nu$. 
\end{definition}

\begin{theorem}
There is a universal commutative formal group law. That is, there is a
commutative FGL $(\bbL,\mu)$ such that if $(R,\nu)$ is any commutative
FGL, there exists a unique morphism $f: (\bbL,\mu)\rightarrow
(R,\nu)$. 
\end{theorem}

\begin{proof}
Let $\bbL=\bbZ[a_{ij} \st i,j \in \bbN ]/I$ where $I$ is a particular
ideal constructed below, and $\mu(x,y)=x+y+\sum a_{ij}x^iy^j$. The
ideal $I$ is taken to be the one generated by:
\begin{enumerate}
\item $a_{ij}-a_{ji}$ to ensure commutativity;
\item $b_{ijk}$ where $\mu(\mu(x,y),z)-\mu(x,\mu(y,z)) =
\sum_{i,j,k\geq 1} b_{ijk}x^iy^jz^k$.
\end{enumerate}
By construction, $\bbL$ with $\mu(x,y)=x+y+\sum a_{ij}x^iy^j$ is a
commutative formal group law. We abuse notation and write $a_{ij}$ for
the class $[a_{ij}] \in \bbL$.

It is also clearly universal. If $(R,\nu)$, $\nu = x+y + \sum
b_{ij}x^iy^j$ is a commutative FGL, there is a unique ring morphism $f
: \bbZ[a_{ij} \st i,j \in \bbN ] \rightarrow R$ where $f(a_{ij}) =
b_{ij}$. Since $\nu$ is a commuatative FGL, all the relations of
$f(I)=0$, and so $f$ factors uniquely through $\tilde{f}: \bbL
\rightarrow R$. This map is a morphism of commutative FGLs, and the
proof is complete.
\end{proof}

\begin{theorem}
There is an isomorphism $\bbL \cong \bbZ[x_2,x_4,x_6,...]$ where
$x_2=a_{11}$, $x_4=a_{12}$, $x_6=a_{22}-a_{13}$, \ldots.
\end{theorem}

\begin{proof} Given in section 7, p. 64. It is a bit messy. 
\end{proof}


\subsection{ FGLs from oriented cohomology theories}

\begin{definition}
A ring spectrum is a spectrum $\bbE$ equipped with maps $m : \bbE
\wedge \bbE \rightarrow \bbE$, $u : \bbS \rightarrow \bbE$ so that the
following diagrams commute. If we are working with symmetric spectra,
the diagrams are to commute on the nose, if we are not working with
symmetric spectra, we need the diagrams to commute in the (stable)
homotopy category of spectra. 

\begin{center}
$\xymatrix{ \bbE \wedge \bbE \wedge \bbE \ar[r]^{m \wedge \id} \ar[d]_{\id \wedge m} & \bbE \wedge \bbE \ar[d]^{m} \\ \bbE \wedge \bbE \ar[r]^{m} & \bbE}$ \hspace{1in}
$\xymatrix{\bbS \wedge \bbE \ar[r]^{u\wedge \id} \ar[dr]_{\pi} & \bbE \wedge \bbE \ar[d]_{m} & \bbE \wedge \bbS \ar[l]_{\id \wedge u} \ar[dl]^{\pi} \\ & \bbE &}$

$\xymatrix{ \bbE \wedge \bbE \ar[r]^{\tau} \ar[rd]_{m} & \bbE \wedge \bbE \ar[d]^{m} \\ & \bbE}$
\end{center}

That is, a ring spectrum is a commutative monoid in the category of
symmetric spectra.
\end{definition}

\begin{definition}
Let $\bbE$ be a commutative ring spectrum (perhaps also an
$\Omega$-spectrum). Observe that a commutative ring spectrum has
canonically defined cohomology classes $[u_n] \in \bbE^n(\bbS^n)$
given by the map $u_n : \bbS^n \rightarrow E_n$. 

An orientation of $\bbE$ is the choice of a generator $x \in
\bbE^2(\bbCP^{\infty})$ so that for $\iota : \bbCP^1
\rightarrow \bbCP^{\infty}$ the standard inclusion, we have
$\iota^*x = [u_2]$.
\end{definition}

Why does this definition give an ``orientation'' to a cohomology
theory? See my comments on this in my Study Guide. 

\begin{theorem}
If $\bbE$ is an oriented cohomology theory, then we can make some
specific cohomology and homology calculations. They are proven in
Adams. The notation $\bbE^*(pt)=\pi_*(\bbE)$ for the coefficients of
the cohomology theory $\bbE$ is used in Adams. For a ring spectrum
$\bbE$, the coefficients $\bbE^*(pt)$ is a ring.
\begin{enumerate}
\item $\bbE^*(\bbCP^n) \cong \bbE^*(pt)[x]/(x^{n+1})$
\item $\bbE^*(\bbCP^{\infty} \cong \bbE^*(pt)[[x]]$ 
\item $\bbE^*( \bbCP^n \times \bbCP^m) \cong \bbE^*(pt) [x_1,x_2] /
(x_1^{n+1},x_2^{n+2})$
\item $\bbE^*(\bbCP^{\infty}\times\bbCP^{\infty}) \cong
\bbE^*(pt)[[x_1,x_2]]$
\end{enumerate}
\end{theorem}

Assuming this theorem, we now construct a FGL for $\bbE$. We are to
think of $x$ as the ``first Chern class'' for our generalized
cohomology theory. With this, we are able to construct a theory of
characteristic classes using the splitting principle. 

There is a map $m: \bbCP^{\infty} \times \bbCP^{\infty}
\rightarrow \bbCP^{\infty}$ which classifies the tensor product of
the universal line bundles $E_1=\pi_1^*E$, $E_2=\pi_2^*E$ over
$\bbCP^{\infty}\times \bbCP^{\infty}$ where $E$ is the universal
line bundle over $\bbCP^{\infty}$ and $\pi_1$ and $\pi_2$ are the
canonical projections. With $m$ chosen in this fashion---it will be
unique up to homotopy by basic classifying space theory---we then
investigate what properties $m^*(x)\in\bbE^*(pt)[[x_1,x_2]]$ has. We
may write $m^*(x) =\mu(x_1,x_2) = x_1 + x_2 + \sum b_{ij}x_1^i x_2^j$
where $b_{ij}\in \bbE^*(pt)$, and it is indeed the case that $\mu$ is
a FGL. How does one see this? Adams says it is easy. First one has to
make sense of what is being asked, then it is quite easy after all.

We need to check a few things. First off, let us make sense of
$\mu(\mu(x_1,x_2),x_3) = \mu(x_1,\mu(x_2,x_3))$. Consider the diagram
\begin{center}
$\xymatrix{
  E_1\otimes E_2 \otimes E_3 \ar[r] \ar[d]_{\xi_1\xi_2\xi_3} & E \ar[d]^{\xi} \\
  \bbCP^{\infty}\times\bbCP^{\infty}\times\bbCP^{\infty}
  \ar[r]^-{M} & \bbCP^{\infty} }$.
\end{center}
It is written ambiguosly without parentheses because they don't matter
in this diagram; they are all equivalent up to
homeomorphism. Therefore $M^*(x) = \mu(\mu(x_1,x_2),x_3) =
\mu(x_1,\mu(x_2,x_3))$. Likewise, we are to interpret $\mu(x,0)=x$ as 
\begin{center}
$\xymatrix{
E_1 \ar[r] \ar[d] &  E \ar[d] \\ 
\bbCP^{\infty}\times\bbCP^{\infty} \ar[r]^-{\pi_1} & \bbCP^{\infty}
}$
\end{center}
so pulling back $x$ under $pi_1^*$ in $\bbE^*(pt)[[x_1,x_2]]$ is just
$x_1$. Likewise the commutativity can be seen. 

This method of getting a formal product is just a different way of
stating the definition in Levine and Morel. Adams shows the usual
topological examples where $\bbE = H\bbZ$ the Eilenberg-MacLane
spectrum where $m^*(x)=x_1+x_2$. One can also take $\bbE = K = BGL$,
the spectrum for $K$-theory, and see that $m^*x = x_1 + x_2 + x_1x_2$. 

Perhaps the other way of writing down the FGL for $\bbE^*$ makes it
more clear how to work with it and verify the properties. The FGL
$\mu(x,y)$ is the element of $\bbE^*(pt)[[x,y]]$ such that for any
line bundles $L_1,L_2$ over $\bbCP^{\infty}$, with classifying maps
$\xi_1,\xi_2 : \bbCP^{\infty} \rightarrow \bbCP^{\infty}$, define
$c_1(L_i)= \xi_i^*(x)$. Then we have
\begin{equation*}
\mu(c_1(L_1),c_1(L_2)) = c_1(L_1\otimes L_2). 
\end{equation*}
In the first definition, it is clear that it exists. In the second
definition, one has to explain why it exists, but it is immediate how
one verifies the FGL properties. 

\begin{theorem}
The FGL associated with the oriented cohomology theory $MU^*$ is the
universal FGL. That is, there is a map $\phi : \bbL \rightarrow
MU^*(pt)$ which is an isomorphism so that the push-forward of the
universal FGL agrees with the FGL on $MU^*$. 
\end{theorem}

This is discussed in section 8 of Adams's blue book.


Levine; Morel. {\it Algebraic Cobordism.}


\section{Model Categories}
\subsection{Generalities}

\begin{enumerate}
\item J. P. May, {\it More Concise Algebraic Topology}
\item Davis \& Kirk, {\it Lecture Notes in Algebraic Topology.}
\item Daniel Quillen, {\it Homotopical Algebra.}
\item Kathryn Hess, {\it Model Categories in Algebraic Topology.}
\end{enumerate}

\subsection{Model structures on particular categories}

\subsection{Cohomology theories}


\section{Algebraic Cobordism}

\section{Topological Cobordism}

\section{Motivic Cohomology}


\section{Complete reference list}
\begin{enumerate}
\item Adams, {\it Stable Homotopy and Generalised Homology}
\item Artin, {\it Grothendieck Topologies.}
\item Blander, {\it Local projective model structures on simplicial presheaves.}
\item Boardman, {\it Stable homotopy theory.}
\item Bott; Tu, {\it Differential Forms in Algebraic Topology.}
\item Bousfield; Friedlander, {\it Homotopy theory of $\Gamma$-spaces, spectra, and bisimplicial sets.}
\item Brown, {\it Cohomology Theories.} 
\item Davis \& Kirk, {\it Lecture Notes in Algebraic Topology.}
\item Hatcher, {\it Algebraic Topology.}
\item Heller, {\it Stable Homotopy Categories.}
\item Hess, {\it Model Categories in Algebraic Topology.}
\item Hilton \& Stammbach, {\it A Course in Homological Algebra.}
\item Hopkins; Quick, {\it Hodge filtered complex bordism.}
\item Hovey, {\it Model categories.}
\item Hovey, {\it Spectra and symmetric spectra in general model categories.}
\item Hovey; Shipley; Smith, {\it Symmetric Spectra.}
\item Hovey; Palmieri; Strickland, {\it Axiomatic Stable Homotopy Theory.}
\item Jardine, {\it Fields Lectures: Presheaves of spectra} (2007).
\item Jardine, {\it Fields Lectures: Simplicial presheaves.}
\item Jardine, {\it Presheaves of symmetric spectra.}
\item Jardine, {\it Stable homotopy theory of simplicial presheaves.}
\item Levine; Morel. {\it Algebraic Cobordism.}
\item Mac Lane, {\it Categories for the Working Mathematician.}
\item May, {\it A Concise Course in Algebraic Topology.}
\item May, {\it More Concise Algebraic Topology}
\item May, {\it Simplicial Objects in Algebraic Topology.}
\item Mazza; Voevodsky; Weibel, {\it Lecture Notes on Motivic Cohomology.}
\item Milnor \& Stasheff {\it Characteristic Classes.}
\item Panin; Pimenov; R\"ondigs, {\it A universality theorem for Voevodsky's algebraic cobordism spectrum.}
\item Panin; Pimenov; R\"ondigs, {\it On the relation of Voevodsky's algebraic cobordism to Quillen's $K$-theory.}
\item Panin; Pimenov; R\"ondigs, {\it On Voevodsky's algebraic $K$-theory spectrum $BGL$.}
\item Puppe, {\it Stabile Homotopietheorie I.}
\item Quillen, {\it Homotopical Algebra.}
\item Quillen, {\it On the formal group laws of unoriented and complex cobordism theory.}
\item Stong, {\it Notes on Cobordism Theory.}
\item Vick, {\it Homology Theory.}
\item Voevodsky, {\it Seattle lectures: $K$-theory and motivic cohomology.}
\item Weibel, {\it An Introduction to Homological Algebra.}
\item Weston, {\it An Introduction to Cobordism Theory.}
\item Whitehead, {\it Generalized Homology Theories.} 
\end{enumerate}
%Abbildung \ref{Badeszene} zeigt...
%--------------------------------------------------------------------------------
%For German stuff: \usepackage[german]{babel}
%--------------------------------------------------------------------------------
%"` or \glqq 	„
%"' or \grqq 	“
%"< or \flqq 	«
%"> or \frqq 	 »
%\flq 	‹
%\frq 	›
%\dq 	"
%--------------------------------------------------------------------------------
%\def\Sample{Unicode - -- ---}
%\ImFellEng \Sample \\
%\Sample ff ffi fl fj fk
%--------------------------------------------------------------------------------
%Use \def\<my name>{my whatever} to recall long bits of repeated text or code
%Use mapping=tex-text to get standard tex lig's in addition to defined lig's
%Use color=ABCDEF to get colored text
%	place above at "Font Name:<___>"
%--------------------------------------------------------------------------------
%Images \setbox0=\hbox{\XeteXpicfile "mypic.jpg"}
%--------------------------------------------------------------------------------
%{\setlength{\baselineskip}{1.5\baselineskip}
%...your text...
%\par}
%--------------------------------------------------------------------------------

\end{document}
